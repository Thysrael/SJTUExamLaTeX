\documentclass[a4paper,12pt]{article}

% `fontset=none` means forbidden default the font setting
\usepackage[UTF8, fontset=none]{ctex} % cjk
\usepackage[margin=0.79in]{geometry} % layout
\usepackage{amsmath, amssymb} % math
\usepackage{fancyhdr} % head and footer
\usepackage{lastpage} % get total page number
\usepackage{xifthen} % enhanced if-else condition
\usepackage{multirow} % merge table item
\usepackage{tabularx} % enhanced table
\usepackage{afterpage} % add table at the second page
\usepackage{tabularray} % enhanced table 2
\usepackage{titlesec} % custom title format
\usepackage{indentfirst} % custom section ident
\usepackage{enumitem} % custom item
% \usepackage{pgffor} % for-loop
% \usepackage{totcount} % counter

% % register counter
% \regtotcounter{section}
% \newcounter{sectioncount}

% font setting
\setmainfont{Times New Roman}
\setmonofont{Consolas}
\setCJKmainfont{SimSun}
\setCJKsansfont{SimHei}
\setCJKmonofont{FangSong}
\setCJKfamilyfont {zhkai} {KaiTi}
\NewDocumentCommand \kaishu{}{\CJKfamily{zhkai}}
% get current semester
\newcommand{\currentsemester}{
  \ifnum\month<10
    1
  \else
    2
  \fi
}

\newcommand{\school}{上海交通大学}
\newcommand{\papertype}{A}
\newcommand{\startyear}{\the\year}
\newcommand{\nextyear}{\the\numexpr\startyear+1\relax}
\newcommand{\semester}{\currentsemester}
\newcommand{\coursename}{操作系统}

% header and footer
\pagestyle{fancy}
\fancyhf{} % clear the default config
% header
\renewcommand{\headrulewidth}{0.4pt}
\renewcommand{\headrule}{\hrule width\textwidth height\headrulewidth}
% footer
\fancyfoot[C]{
  \footnotesize
  \underline{\ \papertype\ } 卷 \space
  第 \underline{\ \thepage\ } 页 \space
  共 \underline{\ \pageref{LastPage}\ } 页
}

% indepent title numbering
\counterwithout{subsection}{section}
\counterwithout{subsubsection}{subsection}

% custom section format
\renewcommand{\thesection}{\chinese{section}}
\titleformat{\section}
  {\normalfont\Large\bfseries}
  {\thesection、}
  {0pt}
  {}

% custom subsection format
\renewcommand{\thesubsection}{\arabic{subsection}}
\titleformat{\subsection}[runin]
  {\normalfont\large\bfseries}
  {\thesubsection.}
  {0pt}
  {}

% custom subsubsection format
% \setcounter{sectioncount}{\number\totvalue{section}}
\renewcommand{\thesubsubsection}{\arabic{subsubsection}}
\titleformat{\subsubsection}[runin]
  {\normalfont\bfseries}
  {\thesubsubsection)\ \ }
  {0pt}
  {}

% subsubsection indent
\titlespacing{\subsubsection}{2em}{1em}{0pt}

% paragraph space
\setlength{\parskip}{0em}

% custom list
\setlist[itemize]{
    itemsep=0em,
    parsep=0.5em,
    topsep=0.5em,
    partopsep=0em
}

\setlist[enumerate]{
    itemsep=0em,
    parsep=0.5em,
    topsep=0.5em,
    partopsep=0em
}

\begin{document}

% title
\begin{center}
  \zihao{4} \school 试卷(\underline{\papertype} 卷)\\
  \zihao{5} (\underline{\ \startyear\ } 至 \underline{\ \nextyear\ } 学年 第 \underline{\ \semester\ } 学期) \\
  \vspace{0.5cm}
  \zihao{5}
  \begin{tabularx}{0.9 \textwidth}{XXX}
    班级号:\underline{\hspace{0.15 \textwidth}} & 学号:\underline{\hspace{0.2 \textwidth}} & 姓名:\underline{\hspace{0.2 \textwidth}} \\
    \multicolumn{2}{l}{课程名称:\underline{\makebox[0.43 \textwidth][l]{\hfill \coursename \hfill}}} & 成绩:\underline{\hspace{0.2 \textwidth}} \\
  \end{tabularx}
\end{center}


% commitment
\afterpage{
  \noindent
  \begin{minipage}{0.22 \textwidth}
    \zihao{5}
    \kaishu \hspace{2em} \textbf{我承诺,我将严格遵守考试纪律。}
    \vspace{2.5em}

    承诺人:\underline{\hspace{0.5 \textwidth}} \\
    \vspace{1em}
  \end{minipage}
  \hfill
  \begin{minipage}{0.73\textwidth}
    \begin{tblr}{
      colspec = {|X[3,l]|X[c]|X[c]|X[c]|X[c]|X[c]|X[c]|X[c]|X[c]|X[c]|},
      rows = {ht=10mm},
      hlines,
      vlines,
      }
      \textbf{题号} & & & & & & & & & \\
      \textbf{得分} & & & & & & & & & \\
      \textbf{批阅人} & & & & & & & & & \\
    \end{tblr}
  \end{minipage}
}


% \foreach \i in {1,2,3,4,5,6,7,8,9} {
% \&
% \ifnum\i>\value{sectioncount} % 比较数值而非计数器
%   0
% \else
%   \i
% \fi
% }
% TODO: 标题样式
% TODO: 分数计算
% TODO: 代码块和图片样式
% TODO: 答案编辑

\section{内存抽象 (36')}

\subsection{}

虚拟内存技术的出现改变了程序员编写应用程序时的基本假设,为用户态程序提供了大量独占连续内存的假象,极大地方便了开发。

\subsubsection{}

TLB 的作用是什么?如果使用单级页表还需要 TLB 吗?(4')

\subsubsection{}

当操作系统选择 4KB 作为最小页大小时, L2 页表项和 L1 页表项对应大页的页面大小分别为 2MB 和 1GB。
那么当操作系统选择 16KB 或 64KB 作为最小页大小时,对应大页(只考虑 L2 页表项)的页面大小是多少?请写出计算过程。(4')


\subsection{}

伙伴系统( buddy system )是一种高效的内存分配器算法,其本质是一种可以快速分裂和合并空闲块的分离空闲链表。(“空闲链表 x ” 表示每一块有 $2^{x}$ 个页)

\subsubsection{}

假设伙伴系统初始只有一块 64KB 大小的内存块,保存在空闲链表 4 当中。依次分配 30KB、9KB、4KB 三次内存后,请问空闲链表 0~3 各剩几个空闲块?(4')

\subsubsection{}

伙伴系统会存在内部碎片的问题,请说明内部碎片是什么,并计算第一问中存在多大的内部碎片。(4')

\section{系统安全 (16')}

\subsection{}

如果允许且只允许你修改一台正在运行中的计算机内存中的任意一个 bit ,你该如何获取到最高权限?简述不少于 2 种方案。(4')

\subsection{}

从安全的角度分析,为什么对文件操作不使用 inode 号作为参数而要用 fd 作为参数?不少于 2 种原因。(4')

\subsection{}

现在当拥有一台 Copilot+ 电脑,用户可以在电脑上开启微软最新推出的 Recall 功能。
Recall 是结合了 Copilot 学习、理解、推理能力应运而生的「回溯」技术,即利用大模型不断追踪用户在 Windows 上的操作,每隔几秒就会给用户的屏幕拍摄一张快照,
在用户需要时,输入一段文字,它将利用过去的任何线索进行搜索,以时间线的形式回放操作,允许用户浏览过去的活动,包括应用、文档和网站。

\subsubsection{}

过去潜在的黑客大费周章却只能针对单一应用攻击,但现在他们或许有机会一次性从你的电脑中获得最近一小时甚至一个月的操作记录,如何避免这样的事情发生?
不少于 3 点(可从系统安全保护层次、文件安全性隔离性保护、应用隔离等方面回答)。(4')

\subsubsection{}

Recall 在拍摄快照时,并不会隐藏密码或重要帐号等信息,它甚至已经记录下你的所有隐私操作。
请你选择课程中介绍的访问控制策略,设计一个允许用户控制的 Recall 启用方案,允许用户决定操作是否允许被 Recall 记录,并确保记录下的数据被安全使用。(4')


\label{LastPage}
\end{document}
