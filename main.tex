\documentclass[a4paper,12pt]{article}

% `fontset=none` means forbidden default the font setting
\usepackage[UTF8, fontset=none]{ctex} % cjk
\usepackage[margin=0.79in]{geometry} % layout
\usepackage{amsmath, amssymb} % math
\usepackage{fancyhdr} % head and footer
\usepackage{lastpage} % get total page number
\usepackage{xifthen} % enhanced if-else condition
\usepackage{multirow} % merge table item
\usepackage{tabularx} % enhanced table
\usepackage{afterpage} % add table at the second page
\usepackage{tabularray} % enhanced table 2
\usepackage{titlesec} % custom title format
\usepackage{indentfirst} % custom section ident
\usepackage{enumitem} % custom item
% \usepackage{pgffor} % for-loop
% \usepackage{totcount} % counter

% % register counter
% \regtotcounter{section}
% \newcounter{sectioncount}

% font setting
\setmainfont{Times New Roman}
\setmonofont{Consolas}
\setCJKmainfont{SimSun}
\setCJKsansfont{SimHei}
\setCJKmonofont{FangSong}
\setCJKfamilyfont {zhkai} {KaiTi}
\NewDocumentCommand \kaishu{}{\CJKfamily{zhkai}}
% get current semester
\newcommand{\currentsemester}{
  \ifnum\month<10
    1
  \else
    2
  \fi
}

\newcommand{\school}{上海交通大学}
\newcommand{\papertype}{A}
\newcommand{\startyear}{\the\year}
\newcommand{\nextyear}{\the\numexpr\startyear+1\relax}
\newcommand{\semester}{\currentsemester}
\newcommand{\coursename}{操作系统}

% header and footer
\pagestyle{fancy}
\fancyhf{} % clear the default config
% header
\renewcommand{\headrulewidth}{0.4pt}
\renewcommand{\headrule}{\hrule width\textwidth height\headrulewidth}
% footer
\fancyfoot[C]{
  \footnotesize
  \underline{\ \papertype\ } 卷 \space
  第 \underline{\ \thepage\ } 页 \space
  共 \underline{\ \pageref{LastPage}\ } 页
}

% indepent title numbering
\counterwithout{subsection}{section}
\counterwithout{subsubsection}{subsection}

% custom section format
\renewcommand{\thesection}{\chinese{section}}
\titleformat{\section}
  {\normalfont\Large\bfseries}
  {\thesection、}
  {0pt}
  {}

% custom subsection format
\renewcommand{\thesubsection}{\arabic{subsection}}
\titleformat{\subsection}[runin]
  {\normalfont\large\bfseries}
  {\thesubsection.}
  {0pt}
  {}

% custom subsubsection format
% \setcounter{sectioncount}{\number\totvalue{section}}
\renewcommand{\thesubsubsection}{\arabic{subsubsection}}
\titleformat{\subsubsection}[runin]
  {\normalfont\bfseries}
  {\thesubsubsection)\ \ }
  {0pt}
  {}

% subsubsection indent
\titlespacing{\subsubsection}{2em}{1em}{0pt}

% paragraph space
\setlength{\parskip}{0em}

% custom list
\setlist[itemize]{
    itemsep=0em,
    parsep=0.5em,
    topsep=0.5em,
    partopsep=0em
}

\setlist[enumerate]{
    itemsep=0em,
    parsep=0.5em,
    topsep=0.5em,
    partopsep=0em
}

\begin{document}

% title
\begin{center}
  \zihao{4} \school 试卷(\underline{\papertype} 卷)\\
  \zihao{5} (\underline{\ \startyear\ } 至 \underline{\ \nextyear\ } 学年 第 \underline{\ \semester\ } 学期) \\
  \vspace{0.5cm}
  \zihao{5}
  \begin{tabularx}{0.9 \textwidth}{XXX}
    班级号:\underline{\hspace{0.15 \textwidth}} & 学号:\underline{\hspace{0.2 \textwidth}} & 姓名:\underline{\hspace{0.2 \textwidth}} \\
    \multicolumn{2}{l}{课程名称:\underline{\makebox[0.43 \textwidth][l]{\hfill \coursename \hfill}}} & 成绩:\underline{\hspace{0.2 \textwidth}} \\
  \end{tabularx}
\end{center}


% commitment
\afterpage{
  \noindent
  \begin{minipage}{0.22 \textwidth}
    \zihao{5}
    \kaishu \hspace{2em} \textbf{我承诺,我将严格遵守考试纪律。}
    \vspace{2.5em}

    承诺人:\underline{\hspace{0.5 \textwidth}} \\
    \vspace{1em}
  \end{minipage}
  \hfill
  \begin{minipage}{0.73\textwidth}
    \begin{tblr}{
      colspec = {|X[3,l]|X[c]|X[c]|X[c]|X[c]|X[c]|X[c]|X[c]|X[c]|X[c]|},
      rows = {ht=10mm},
      hlines,
      vlines,
      }
      \textbf{题号} & & & & & & & & & \\
      \textbf{得分} & & & & & & & & & \\
      \textbf{批阅人} & & & & & & & & & \\
    \end{tblr}
  \end{minipage}
}


% \foreach \i in {1,2,3,4,5,6,7,8,9} {
% \&
% \ifnum\i>\value{sectioncount} % 比较数值而非计数器
%   0
% \else
%   \i
% \fi
% }
% TODO: 标题样式
% TODO: 分数计算
% TODO: 代码块和图片样式
% TODO: 答案编辑

\section{系统安全(20')}

\subsection{}

Private Cloud Compute (PCC) 是 Apple 提出的云 AI 计算安全架构,基于用户的输入和数据进行推理。为了保护用户的隐私,PCC 采取了一系列的安全措施:

\begin{itemize}
    \item 基于苹果自研芯片(如 M2 Ultra/M4),具有不可被篡改的密钥存储区域;
    \item PCC Server 运行专有的操作系统;关键安全组件大量使用内存安全编程语言(如 Swift, Rust)编写,减少内存破坏类漏洞的风险。
    \item 禁用远程 Shell 和调试工具,仅允许结构化日志输出;
    \item 采用随机映射的方式,让数据中心的 PCC Server 与用户之间的映射呈现随机的模式
    \item 系统启动时,从硬件信任根开始,每一步加载的固件都必须经过密码学验证签名,确保只加载和执行苹果授权的、未被篡改的代码。
\end{itemize}

\subsubsection{}

相比所有 AI 推理均在本地执行,PCC 架构所面临的安全威胁有哪些?哪些威胁可以被上述技术解决,哪些不可以?(4')


\subsubsection{}

用户数据最终会在 PCC 服务器解密后用于推理。如何才能实现用户数据的全生命周期保护(包括内存中)? (4')

\section{第三节}
内容...

\section{第四节}
内容...

再见,世界。

This is good!

\verb|printf("b")|

\newpage

\section{第五节}
内容...

\section{第六节}
内容...

\verb|printf("b")|

\label{LastPage}
\end{document}
